\documentclass[11pt,a4paper]{article}
\usepackage{amsmath}
\usepackage{amsthm}
\usepackage{amssymb}
\usepackage[margin=2cm]{geometry}
%\usepackage{thmbox}
\usepackage{graphicx}
\usepackage[dvipsnames,usenames]{color}
\usepackage{url}
\usepackage{comment}
\usepackage{amsmath, amsthm, amssymb,enumerate}

%\usepackage{enumerate}
%\usepackage{titlesec}
%\usepackage{Rvector}
%\usepackage{mathabx}
\newcommand{\qrq}{\quad\Rightarrow\quad}
\newcommand{\qarq}{\quad&\Rightarrow\quad}
\newcommand{\alp}{\alpha}
\newcommand{\claim}{{\underline{\it Claim:}}~~}
\newcommand{\dbR}{\mathbb{R}}
\newcommand{\ndimr}{\mathbb{R}^n}
\newcommand{\vare}{\varepsilon}
\newcommand{\since}{\because\;}
\newcommand{\hence}{\therefore\;}
\newcommand{\en}{\par\noindent}
\newcommand{\fn}{\footnotesize}

\newcommand{\sect}[2]{#1~~{\mdseries\tiny(#2)}}

\renewcommand{\(}{\left(}
\renewcommand{\)}{\right)}

\let \ds=\displaystyle

\usepackage{xeCJK}
\setCJKmainfont[AutoFakeBold=5,AutoFakeSlant=.4]{標楷體}

%\usepackage{fancyhdr}
%\pagestyle{fancy}
%\renewcommand{\headrulewidth}{0pt}

\renewcommand{\thesection}{Lecture \arabic{section}}
\renewcommand{\thesubsection}{\Roman{subsection}}

\usepackage[T1]{fontenc}

%%%% F U N C T I O N %%%%%
\newcommand{\abs}[1]{\left|#1\right|}
\newcommand{\norm}[1]{\left\|#1\right\|}
\newcommand{\inn}[1]{\left<#1\right>}
\newcommand{\f}[1]{f\!\left(#1\right)}
\newcommand{\g}[1]{g\!\left(#1\right)}
\newcommand{\h}[1]{h\!\left(#1\right)}
\newcommand{\x}[1]{x\!\left(#1\right)}
\newcommand{\D}[1]{D\!\left(#1\right)}
\newcommand{\N}[1]{N\!\left(#1\right)}
\renewcommand{\P}[1]{P\!\left(#1\right)}
\newcommand{\R}[1]{R\!\left(#1\right)}
\newcommand{\V}[1]{V\!\left(#1\right)}
\newcommand{\function}[2]{#1\!\left(#2\right)}
\newcommand{\functions}[2]{\left(#1\right)\!\left(#2\right)}

\definecolor{light-gray}{gray}{0.95}
\newcommand{\textfil}[1]{\colorbox{light-gray}{\large\color{Red} #1}}


\renewcommand{\title}{Numerical Optimization with applications: Homework 07}
\renewcommand{\author}{104021601 林俊傑\\104021602 吳彥儒\\104021615 黃翊軒}
\renewcommand{\maketitle}{\begin{center}\textbf{\Large\title}\\[6pt] {\author}\\[6pt] {\color{Gray}\footnotesize December 14, 2016}\end{center}}
\newcommand{\blue}[1]{{\color{blue}#1}}


\renewcommand{\labelenumi}{(\alph{enumi})}

\newcommand{\Exercise}[2]{\textbf{Exercise #1.} \textit{#2}}
\newtheorem{exercise}{Exercise}

%\parskip=11pt

\begin{document}

  \maketitle
  \begin{proof}
  Since all norms in $\mathbb{R}^n$ are equivalent.
  \[ \exists \alpha>0 \quad \mbox{such that} \quad \norm{x} \leq \alpha \norm{x}_{\infty} \quad \forall x \in \mathbb{R}^n \]
  We have,\\
  \begin{align*}  
   \norm{J(x_1)-J(x_2)} &= \max_{\norm{y}=1} \norm{(J(x_1)-J(x_2))y}\\
   &= \max_{\norm{y}=1} \norm{
	\begin{bmatrix}
	(\bigtriangledown r_1(x_1)-\bigtriangledown r_1(x_2))^Ty \\
	\vdots\\
	(\bigtriangledown r_m(x_1)-\bigtriangledown r_m(x_2))^Ty
	\end{bmatrix}	   
   }\\
   &\leq \max_{\norm{y}=1} \alpha \norm{
	\begin{bmatrix}
	(\bigtriangledown r_1(x_1)-\bigtriangledown r_1(x_2))^Ty \\
	\vdots\\
	(\bigtriangledown r_m(x_1)-\bigtriangledown r_m(x_2))^Ty
	\end{bmatrix}	   
   }_{\infty}\\
   &= \alpha \max_{\norm{y}=1}~\max_{1 \leq j \leq m} |(\bigtriangledown r_j(x_1)-\bigtriangledown r_j(x_2))^Ty|\\
   &\leq \alpha \max_{\norm{y}=1}~\max_{1 \leq j \leq m} |(\bigtriangledown r_j(x_1)-\bigtriangledown r_j(x_2))|~|y|\\
   &\leq \alpha \max_{\norm{y}=1}~\max_{1 \leq j \leq m} L\norm{x_1-x_2}~|y|\\
   &= \alpha L \norm{x_1-x_2}
  \end{align*}
  We conclude that $J$ is Lipschitz continuous with constant $\tilde{L}=\alpha L$.
  \end{proof}
  
%exercise10.2 part(b)
\begin{proof}
	Given $x,\tilde{x}$ in $\mathcal{D}$, we estimate
	\begin{align*}
	\|\nabla f(x)-\nabla f(\tilde{x})\| &= \|J(x)^Tr(x)-J(\tilde{x})^Tr(\tilde{x})\|\\
	&= \|\left[J(x)^Tr(x)-J(\tilde{x})^Tr(x)\right] + \left[J(\tilde{x})^Tr(x)-J(\tilde{x})^Tr(\tilde{x})\right]\|\\
	&= \|\left(J(x)^T-J(\tilde{x})^T\right)r(x) + J(\tilde{x})^T\left(r(x)-r(\tilde{x})\right)\|\\
	& \le \|J(x)^T-J(\tilde{x})^T\||r(x)|+\|J(\tilde{x})^T\|\|r(x)-r(\tilde{x})\|\\
	&\le M\alpha L \|x-\tilde{x}\|+M'L\|x-\tilde{x}\|\\
	&= \mathcal{L}\|x-\tilde{x}\|
	\end{align*}
	where $\mathcal{L} =  M\alpha L+M'L$ and $\|J(\tilde{x})^T\|$ is bounded since it is Lipschitz continuous on a compact set $\mathcal{D}$.
	
\end{proof}
\end{document} 