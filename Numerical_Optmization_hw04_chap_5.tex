\documentclass[11pt,a4paper]{article}
\usepackage{amsmath}
\usepackage{amsthm}
\usepackage{amssymb}
\usepackage[margin=2cm]{geometry}
%\usepackage{thmbox}
\usepackage{graphicx}
\usepackage[dvipsnames,usenames]{color}
\usepackage{url}
\usepackage{comment}
\usepackage{amsmath, amsthm, amssymb,enumerate}

%\usepackage{enumerate}
%\usepackage{titlesec}
%\usepackage{Rvector}
%\usepackage{mathabx}
\newcommand{\qrq}{\quad\Rightarrow\quad}
\newcommand{\qarq}{\quad&\Rightarrow\quad}
\newcommand{\alp}{\alpha}
\newcommand{\claim}{{\underline{\it Claim:}}~~}
\newcommand{\dbR}{\mathbb{R}}
\newcommand{\ndimr}{\mathbb{R}^n}
\newcommand{\vare}{\varepsilon}
\newcommand{\since}{\because\;}
\newcommand{\hence}{\therefore\;}
\newcommand{\en}{\par\noindent}
\newcommand{\fn}{\footnotesize}

\newcommand{\sect}[2]{#1~~{\mdseries\tiny(#2)}}

\renewcommand{\(}{\left(}
\renewcommand{\)}{\right)}

\let \ds=\displaystyle

\usepackage{xeCJK}
\setCJKmainfont[AutoFakeBold=5,AutoFakeSlant=.4]{標楷體}

%\usepackage{fancyhdr}
%\pagestyle{fancy}
%\renewcommand{\headrulewidth}{0pt}

\renewcommand{\thesection}{Lecture \arabic{section}}
\renewcommand{\thesubsection}{\Roman{subsection}}

\usepackage[T1]{fontenc}

%%%% F U N C T I O N %%%%%
\newcommand{\abs}[1]{\left|#1\right|}
\newcommand{\norm}[1]{\left\|#1\right\|}
\newcommand{\inn}[1]{\left<#1\right>}
\newcommand{\f}[1]{f\!\left(#1\right)}
\newcommand{\g}[1]{g\!\left(#1\right)}
\newcommand{\h}[1]{h\!\left(#1\right)}
\newcommand{\x}[1]{x\!\left(#1\right)}
\newcommand{\D}[1]{D\!\left(#1\right)}
\newcommand{\N}[1]{N\!\left(#1\right)}
\renewcommand{\P}[1]{P\!\left(#1\right)}
\newcommand{\R}[1]{R\!\left(#1\right)}
\newcommand{\V}[1]{V\!\left(#1\right)}
\newcommand{\function}[2]{#1\!\left(#2\right)}
\newcommand{\functions}[2]{\left(#1\right)\!\left(#2\right)}

\definecolor{light-gray}{gray}{0.95}
\newcommand{\textfil}[1]{\colorbox{light-gray}{\large\color{Red} #1}}


\renewcommand{\title}{Numerical Optimization with applications: Homework 04}
\renewcommand{\author}{104021601 林俊傑\\104021602 吳彥儒\\104021615 黃翊軒}
\renewcommand{\maketitle}{\begin{center}\textbf{\Large\title}\\[6pt] {\author}\\[6pt] {\color{Gray}\footnotesize November 9, 2016}\end{center}}
\newcommand{\blue}[1]{{\color{blue}#1}}


\renewcommand{\labelenumi}{(\alph{enumi})}

\newcommand{\Exercise}[2]{\textbf{Exercise #1.} \textit{#2}}
\newtheorem{exercise}{Exercise}

%\parskip=11pt

\begin{document}

  \maketitle
  
  \begin{exercise}
  	
  \end{exercise}  
  \begin{proof}
  	
  \end{proof}
  
  
  \begin{exercise}
  	
  \end{exercise}  
  \begin{proof}

  \end{proof}
  
  \setcounter{exercise}{3}
  
  \begin{exercise}
  	Show that if $f(x)$ is a strictly convex quadratic, then the function $h(\sigma)\overset{\scriptscriptstyle def}{=}f(x_0+\sigma_0p_0+\cdots+\sigma_{k-1}p_{k-1})$ also is a strictly convex quadratic in the variable $\sigma = (\sigma_0,\sigma_1,\cdots,\sigma_{k-1})^T.$
  \end{exercise}  
  \begin{proof}
	By the definition of strictly convex quadratic function, we can assume $$f(x) = \frac{1}{2}x^TAx-b^Tx,$$ where $A$ is a positive definite symmetric matrix and $b$ is a constant vector.
	We want prove that $h(\sigma)$ is also a strictly convex quadratic function by showing $$h(\sigma) = \frac{1}{2}\sigma^TB\sigma-c^T\sigma+d,$$ where $B$ is a positive definite symmetric matrix and $c,d$ are constant vectors.
	Since $p_i^TAp_j = 0$ for all $i\neq j$, we obtain that
	\begin{align*}
		h(\sigma) &= f(x_0+\sigma_0p_0+\cdots+\sigma_{k-1}p_{k-1})\\
		&=\frac{1}{2}(x_0+\sigma_0p_0+\cdots+\sigma_{k-1}p_{k-1})^TA(x_0+\sigma_0p_0+\cdots+\sigma_{k-1}p_{k-1})-b^T(x_0+\sigma_0p_0+\cdots+\sigma_{k-1}p_{k-1})\\
		&=\frac{1}{2}x_0^TA(x_0+\sigma_0p_0+\cdots+\sigma_{k-1}p_{k-1})+\frac{1}{2}(\sigma_0p_0)^TA(x_0+\sigma_0p_0+\cdots+\sigma_{k-1}p_{k-1})+\cdots \\ &\qquad+\frac{1}{2}(\sigma_{k-1}p_{k-1})^TA(x_0+\sigma_0p_0+\cdots+\sigma_{k-1}p_{k-1})-b^T(x_0+\sigma_0p_0+\cdots+\sigma_{k-1}p_{k-1})\\
		&= \frac{1}{2}(\sigma_0p_0)^TA(\sigma_0p_0)+\frac{1}{2}(\sigma_1p_1)^TA(\sigma_1p_1)+\cdots + \frac{1}{2}(\sigma_{k-1}p_{k-1})^TA(\sigma_{k-1}p_{k-1})\\
		&\qquad+\frac{1}{2}x_0^TA(x_0+\sigma_0p_0+\cdots+\sigma_{k-1}p_{k-1})-b^T(x_0+\sigma_0p_0+\cdots+\sigma_{k-1}p_{k-1})\\
		&= \frac{1}{2}\sigma^T B \sigma+\frac{1}{2}x_0^TAP\sigma-b^TP\sigma+\frac{1}{2}x_0^TAx_0-b^Tx_0\\
		&= \frac{1}{2}\sigma^T B \sigma+ (\frac{1}{2}x_0^TAP-b^TP)\sigma+\frac{1}{2}x_0^TAx_0-b^Tx_0
	\end{align*}
	where $B = 
	\begin{bmatrix}
		p_0^TAp_0 & \cdots & 0\\
		\vdots & \ddots & \vdots\\
		0& \cdots & p_{k-1}^TAp_{k-1}\\
	\end{bmatrix}
	$ is positive definite sysmmetric matrix and $P = (p_0,p_1,\cdots, p_{k-1})$ and $\sigma = (\sigma_0,\sigma_1,\cdots,\sigma_{k-1})^T.$
	Hence, $h(\sigma)$ is also a strictly convex quadratic function.
  \end{proof}
  
  \setcounter{exercise}{6}
  
  \begin{exercise}
  Let $\lbrace \lambda_i,v_i \rbrace ~ i=1,2,\cdots,n$ be the eigenpairs of the symmetric matrix A. Show that the eigenvalues and eigenvectors of $[I+P_k(A)A]^TA[I+P_k(A)A]$ are $\lambda_i[1+\lambda_iP_k(\lambda_i)]^2$ and $v_i$, respectively.
  \end{exercise}  
  \begin{proof}
 We first show that
 \[ P_k(A)v_i = P_k(\lambda_i)v_i\]
 for any polynomials $P_k(x)$ of degree $k$.\\
 Let $\displaystyle P_k(x) = \sum_{j=0}^k a_jx^j$. Then\\
 $\displaystyle P_k(A)v_i=\sum_{j=0}^k a_jA^jv_i=\sum_{j=0}^k a_j A^{j-1}(\lambda_iv_i)=\sum_{j=0}^k a_j A^{j-2} (\lambda_i^2v_i) = \cdots =\sum_{j=0}^k a_j \lambda_i^jv_i = P_k(\lambda_i)v_i $\\
 Since $[I+P_k(x)x]$ is a ploynomial, we have
 \[ [I+P_k(A)A]v_i = [1+\lambda_iP_k(\lambda_i)]v_i\]
 $A$ is symmetric, therefore, $[I+P_k(A)A]^T=[I+P_k(A)A]$\\
 Now,we are ready to compute\\
 \begin{align*}
 [I+P_k(A)A]^TA[I+P_k(A)A]v_i &=[I+P_k(A)A]A[I+P_k(A)A]v_i\\
 &=[I+P_k(A)A]A[1+\lambda_iP_k(\lambda_i)]v_i\\
 &=[I+P_k(A)A](Av_i)[1+\lambda_iP_k(\lambda_i)]\\
 &=[I+P_k(A)A]\lambda_iv_i[1+\lambda_iP_k(\lambda_i)]\\
 &=[I+P_k(A)A]v_i\lambda_i[1+\lambda_iP_k(\lambda_i)]\\
 &=[1+\lambda_iP_k(\lambda_i)]v_i\lambda_i[1+\lambda_iP_k(\lambda_i)]\\
 &=\lambda_i[1+\lambda_iP_k(\lambda_i)]^2 v_i
 \end{align*}
 We conclude that $\lbrace \lambda_i[1+\lambda_iP_k(\lambda_i)]^2,v_i \rbrace ~ i=1,2,\cdots,n$
 are the eigenpairs of \\
 $[I+P_k(A)A]^TA[I+P_k(A)A]$.
\end{proof}
  
\end{document} 